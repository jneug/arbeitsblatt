
%%%%%%%%%%%%%%%%%%%%%%%%%%%%%%%%
%           Optionen           %
%%%%%%%%%%%%%%%%%%%%%%%%%%%%%%%%
\pgfkeys{
	/absetup/.cd,
	anmerkungen als boxen/.value forbidden,
	anmerkungen als boxen/.code=\ab@renewAnmerkungen
}


%%%%%%%%%%%%%%%%%%%%%%%%%%%%%%%%
%            Pakete            %
%%%%%%%%%%%%%%%%%%%%%%%%%%%%%%%%
\RequirePackage{awesomebox}
\PassOptionsToPackage{framemethod=TikZ}{mdframed}
\RequirePackage{mdframed}
\usetikzlibrary{shadows}

%%%%%%%%%%%%%%%%%%%%%%%%%%%%%%%%
%            Makros            %
%%%%%%%%%%%%%%%%%%%%%%%%%%%%%%%%

% Verschiedene Boxen für Hervorhebungen
\newmdenv[
	default,
	linewidth=2pt,
	linecolor=secondary,
]{rahmen}

\newmdenv[
	default,
	linewidth=2pt,
	linecolor=black,
	backgroundcolor=white,
	shadow=true,
	shadowsize=5pt,
	shadowcolor=black!50,
]{schattenbox}

\newmdenv[
	default,
	linewidth=1pt,
	linecolor=black,
	backgroundcolor=primary.hg,
	shadow=true,
	shadowsize=4pt,
	shadowcolor=black!50,
	roundcorner=4,
]{infobox}

% Anmerkungen als Boxen setzen (sofern Modul "typo" geladen wurde)
\newcommand{\ab@renewAnmerkungen}{%
	\ifdef{\tipp}{\renewcommand}{\newcommand}{\tipp}[1]{\tipbox{##1}}
	\ifdef{\hinweis}{\renewcommand}{\newcommand}{\hinweis}[1]{\notebox{##1}}
	\ifdef{\warnung}{\renewcommand}{\newcommand}{\warnung}[1]{\warningbox{##1}}
	\ifdef{\frage}{\renewcommand}{\newcommand}{\frage}[1]{\awesomebox[green!70!red]{2pt}{\faQuestionCircle[regular]}{green!70!red}{##1}}
}
