

%%%%%%%%%%%%%%%%%%%%%%%%%%%%%%%%
%           Optionen           %
%%%%%%%%%%%%%%%%%%%%%%%%%%%%%%%%
\newtoggle{ab@ew@anzeigen}
\newtoggle{ab@nv@anzeigen}
\toggletrue{ab@nv@anzeigen}
\newtoggle{ab@nv@rel}
\toggletrue{ab@nv@rel}

\def\namenKmk{
	15 = 15, 14 = 14, 13 = 13,
	12 = 12, 11 = 11, 10 = 10,
	 9 =  9,  8 =  8,  7 =  7,
	 6 =  6,  5 =  5,  4 =  4,
	 3 =  3,  2 =  2,  1 =  1,
	 0 =  0
}
\edef\kurznamenKmk{\namenKmk}
\def\verteilungKmk{
	15 = .95, 14 = .90, 13 = .85,
	12 = .80, 11 = .75, 10 = .7,
	 9 = .65,  8 = .60,  7 = .55,
	 6 = .50,  5 = .45,  4 = .40,
	 3 = .33,  2 = .27,  1 = .2,
	 0 = .0
}

\def\verteilungNormalMitTendenzen{
	15 = .95, 14 = .90, 13 = .85,
	12 = .80, 11 = .75, 10 = .7,
	 9 = .65,  8 = .60,  7 = .55,
	 6 = .50,  5 = .45,  4 = .40,
	 3 = .33,  2 = .27,  1 = .2,
	 0 = .0
}
\def\namenNormalMitTendenzen{
	15 = {sehr gut\,$+$},
	14 = {sehr gut},
	13 = {sehr gut\,$-$},
	12 = {gut\,$+$},
	11 = {gut},
	10 = {gut\,$-$},
	 9 = {befriedigend\,$+$},
	 8 = {befriedigend},
	 7 = {befriedigend\,$-$},
	 6 = {ausreichend\,$+$},
	 5 = {ausreichend},
	 4 = {ausreichend\,$-$},
	 3 = {mangelhaft\,$+$},
	 2 = {mangelhaft},
	 1 = {mangelhaft\,$-$},
	 0 = {ungenügend}
}
\def\kurznamenNormalMitTendenzen{
	15 = {1$+$},
	14 = {1},
	13 = {1$-$},
	12 = {2$+$},
	11 = {2},
	10 = {2$-$},
	 9 = {3$+$},
	 8 = {3},
	 7 = {3$-$},
	 6 = {4$+$},
	 5 = {4},
	 4 = {4$-$},
	 3 = {5$+$},
	 2 = {5},
	 1 = {5$-$},
	 0 = {6}
}

\def\verteilungNormalMitTendenzen{
	15 = .95, 14 = .90, 13 = .85,
	12 = .80, 11 = .75, 10 = .7,
	 9 = .65,  8 = .60,  7 = .55,
	 6 = .50,  5 = .45,  4 = .40,
	 3 = .33,  2 = .27,  1 = .2,
	 0 = .0
}
\def\namenNormalMitTendenzen{
	15 = {sehr gut\,$+$},
	14 = {sehr gut},
	13 = {sehr gut\,$-$},
	12 = {gut\,$+$},
	11 = {gut},
	10 = {gut\,$-$},
	 9 = {befriedigend\,$+$},
	 8 = {befriedigend},
	 7 = {befriedigend\,$-$},
	 6 = {ausreichend\,$+$},
	 5 = {ausreichend},
	 4 = {ausreichend\,$-$},
	 3 = {mangelhaft\,$+$},
	 2 = {mangelhaft},
	 1 = {mangelhaft\,$-$},
	 0 = {ungenügend}
}
\def\kurznamenNormalMitTendenzen{
	15 = {1$+$},
	14 = {1},
	13 = {1$-$},
	12 = {2$+$},
	11 = {2},
	10 = {2$-$},
	 9 = {3$+$},
	 8 = {3},
	 7 = {3$-$},
	 6 = {4$+$},
	 5 = {4},
	 4 = {4$-$},
	 3 = {5$+$},
	 2 = {5},
	 1 = {5$-$},
	 0 = {6}
}

\def\verteilungNormalOhneTendenzen{
	 5 = .85, 4 = .7,
	 3 = .55, 2 = .40,
	 1 = .20, 0 = .0
}
\def\namenNormalOhneTendenzen{
	 5 = {sehr gut},
	 4 = {gut},
	 3 = {befriedigend},
	 2 = {ausreichend},
	 1 = {mangelhaft},
	 0 = {ungenügend}
}
\def\kurznamenNormalOhneTendenzen{
	 5 = {1},
	 4 = {2},
	 3 = {3},
	 2 = {4},
	 1 = {5},
	 0 = {6}
}


\pgfkeys{
	/absetup/.cd,
		teilpunkte/anzeigen/.code = \settoggle{ab@tpkt@anzeigen}{#1},
		erwartungshorizont/.style = {erwartungshorizont/anzeigen = true},
		erwartungshorizont/anzeigen/.code = \settoggle{ab@ew@anzeigen}{#1},
		erwartungshorizont/stil/.code = \xsimsetup{grading-table/template=#1},
		%
		noten/.style = {noten/#1},
		noten/mit-tendenzen/.style = {
			noten/namen      = \namenNormalMitTendenzen,
			noten/kurznamen  = \kurznamenNormalMitTendenzen,
			noten/verteilung = \verteilungNormalMitTendenzen,
		},
		noten/ohne-tendenzen/.style = {
			noten/namen      = \namenNormalOhneTendenzen,
			noten/kurznamen  = \kurznamenNormalOhneTendenzen,
			noten/verteilung = \verteilungNormalOhneTendenzen,
		},
		noten/kmk-punkte/.style = {
			noten/namen      = \namenKmk,
			noten/kurznamen  = \kurznamenKmk,
			noten/verteilung = \verteilungKmk,
		},
		%
		noten/stil/.code = {},
		noten/anzeigen/.code = \settoggle{ab@nv@anzeigen}{#1},
		noten/relativ/.code = \settoggle{ab@nv@rel}{#1},
		noten/namen/.code = \expandafter\NotennamenFestlegen\expandafter{#1},
		noten/kurznamen/.code = \expandafter\NotenkurznamenFestlegen\expandafter{#1},
		noten/verteilung/.code = {%
			\ab@clearGradingTable%
			\NotenverteilungFestlegen{#1}%
		}
}

%%%%%%%%%%%%%%%%%%%%%%%%%%%%%%%%
%            Pakete            %
%%%%%%%%%%%%%%%%%%%%%%%%%%%%%%%%
\RequirePackage{expl3}
\ab@modul@laden{tabellen}
\ab@modul@laden{aufgaben}


%%%%%%%%%%%%%%%%%%%%%%%%%%%%%%%%
%            Makros            %
%%%%%%%%%%%%%%%%%%%%%%%%%%%%%%%%

%% Teilaufgaben zu Aufgaben hinzufügen
%\DeclareExerciseProperty{erwartungen}
%\DeclareExerciseProperty{teilaufgaben}

\ExplSyntaxOn
%% Utilities
\cs_new:Npn \ab_aufg_id:n #1 {
	aufg#1
}
\cs_new:Npn \ab_taufg_id:nn #1#2 {
	\ab_aufg_id:n {#1}-taufg#2
}
\cs_new:Npn \ab__aufg_seq_cs:n #1 {
	g__ab_\ab_aufg_id:n {#1}_taufg_seq
}
\cs_new:Npn \ab__taufg_ew_seq_cs:n #1 {
	g__ab_#1_ew_seq
}

%% Temp Variables
\tl_new:N 	\l__ab_tmpa_tl
\tl_new:N 	\l__ab_tmpb_tl

\seq_new:N 	\l__ab_tmpa_seq
\seq_new:N 	\l__ab_tmpb_seq

\int_new:N 	\l__ab_tmpa_int
\int_new:N 	\l__ab_tmpb_int

\str_new:N 	\l__ab_tmpa_str
\str_new:N 	\l__ab_tmpb_str

\seq_new:N	\l__ab_tmpa_rows_seq
\seq_new:N	\l__ab_tmpb_rows_seq

%% Variables
% Notennamen
\prop_new:N \l__ab_grade_names_prop
\prop_new:N \l__ab_grade_shortnames_prop

%% Data storage
% 1: xsim-ID Aufgabe
% 2: label Teilaufgabe
% 3: Nummer Teilaufgabe
\cs_new:Npn \ab_aufg_put_taufg:nnn #1#2#3 {
	\tl_set:Nx \l__ab_tmpb_tl {\ab__aufg_seq_cs:n {#1}}
	\seq_if_exist:cTF {\l__ab_tmpb_tl}{}{
		\seq_gclear_new:c {\l__ab_tmpb_tl}
	}
	\seq_gput_right:cx {\l__ab_tmpb_tl} {{#2}{#3}}
}

% 1: ID Teilaufgabe
% 2: Erwartung
% 3: Punkte
\cs_new:Npn \ab_taufg_put_ew:nnn #1#2#3 {
	\tl_set:Nx \l__ab_tmpb_tl {\ab__taufg_ew_seq_cs:n {#1}}
	\seq_if_exist:cTF {\l__ab_tmpb_tl}{}{
		\seq_gclear_new:c {\l__ab_tmpb_tl}
	}
	\seq_gput_right:cx {\l__ab_tmpb_tl} {{#2}{#3}}
}

%% Noten
% Prozentuale untere Schwelle der Note
% 1: Note (Zahl)
% 2: Suffix, wenn Note definiert
% 3: Text, wenn Note nicht definiert
\cs_new_protected:Npn \xsim_get_relative_grade_goal:nnn #1#2#3 {
	\prop_get:NnNTF \l__xsim_relative_grades_prop {#1} \l__xsim_tmpa_tl {
		\fp_to_decimal:n {round( 100*(\l__xsim_tmpa_tl),0)}#2
	}{ #3 }
}

% Definition der Notennamen
% 1: declare-Kommando für Kurz- oder Langnamen
% 2: Kommaliste mit Note=Name Zuordnung
\cs_new_protected:Npn \ab_declare_grade_names:Nn #1#2 {
	\seq_set_split:NVn \l__ab_tmpa_seq \l__xsim_grade_split_tl {#2}
  \seq_map_inline:Nn \l__ab_tmpa_seq {
		#1 ##1 \q_stop
	}
}
% Speicherung eines Langnamen
\cs_new_protected:Npn \__ab_declare_grade_name:w #1 = #2 \q_stop {
	\tl_set:Nx \l__ab_tmpa_tl { \tl_trim_spaces:n {#1} }
	\tl_set:Nx \l__ab_tmpb_tl { \tl_trim_spaces:n {#2} }
	\prop_put:NVV \l__ab_grade_names_prop
		\l__ab_tmpa_tl
		\l__ab_tmpb_tl
}
% Speicherung eines Kursnamens
\cs_new_protected:Npn \__ab_declare_grade_shortname:w #1 = #2 \q_stop {
	\tl_set:Nx \l__ab_tmpa_tl { \tl_trim_spaces:n {#1} }
	\tl_set:Nx \l__ab_tmpb_tl { \tl_trim_spaces:n {#2} }
	\prop_put:NVV \l__ab_grade_shortnames_prop
		\l__ab_tmpa_tl
		\l__ab_tmpb_tl
}

\NewDocumentEnvironment {erwartungen} {}
	{}
	{}
\NewDocumentCommand \erwartung { o m m }{%
	\IfNoValueTF{#1}
		{\tl_set:Nx \l__ab_tmpa_tl {\ab_taufg_id:nn {\GetExerciseProperty{ID}} {0}}}
		{\tl_set:Nx \l__ab_tmpa_tl {\ab_taufg_id:nn {\GetExerciseProperty{ID}} {#1}}}
	\ab_taufg_put_ew:nnn {\l__ab_tmpa_tl} {#2} {#3}
	\addpoints*{#3}%
}

\RenewDocumentEnvironment {aufgabenteil} { s o } {\teilaufgaben%
	% \refstepcounter{teilaufgabe}%
	\item%
	\IfNoValueTF{#2}{%
		\tl_set:Nx \l__ab_tmpa_tl {\ab_taufg_id:nn {\GetExerciseProperty{ID}} {\alph{teilaufgabeni}}}%
	}{%
		\tl_set:Nx \l__ab_tmpa_tl {\ab_taufg_id:nn {\GetExerciseProperty{ID}} {#2}}%
	}%
	\ab_aufg_put_taufg:nnn {\GetExerciseProperty{ID}} {\l__ab_tmpa_tl} {\theteilaufgabe}%
	%\theteilaufgabe\label{taufg:\l__ab_tmpa_tl}
	\label{taufg:\l__ab_tmpa_tl}
}{%
	\endteilaufgaben
}
\RenewDocumentCommand \teilaufgabe {s o} {%
	\item%
	\IfNoValueTF{#2}{%
		\tl_set:Nx \l__ab_tmpa_tl {\ab_taufg_id:nn {\GetExerciseProperty{ID}} {\alph{teilaufgabeni}}}%
	}{%
		\tl_set:Nx \l__ab_tmpa_tl {\ab_taufg_id:nn {\GetExerciseProperty{ID}} {#2}}%
	}%
	\label{taufg:\l__ab_tmpa_tl}%
	\ab_aufg_put_taufg:nnn {\GetExerciseProperty{ID}} {\l__ab_tmpa_tl} {\theteilaufgabeni}%
}

\def\tabularxcolumn#1{m{#1}}
\newcolumntype{M}[1]{>{\large\centering\arraybackslash}m{#1}}


\cs_new:Npn \ab_punkte_gesamt: {
	\xsim_print_goal_sum:nnnn {aufgabe} {points} {} {}
}
\cs_new:Npn \ab_bonus_gesamt: {
	\xsim_print_goal_sum:nnnn {aufgabe} {bonus-points} {} {}
}
\cs_new:Npn \ab_punkte_total: {
	\xsim_print_goals_sum:nnnn {aufgabe} {points+bonus-points} {} {}
}
\cs_new:Npn \ab_punkte_aufg:n #1 {
	\xsim_get_property:nnn {aufgabe} {#1} {points}
}
\cs_new:Npn \ab_punkte_taufg:nn #1#2 {
	\tl_set:Nx \l__ab_tmpa_tl {\ab_taufg_id:nn {#1} {#2}}
	\ab_punkte_taufg:n \l__ab_tmpa_tl
}
\cs_new:Npn \ab_punkte_taufg:n #1 {
	\tl_set:Nx \l__ab_tmpb_tl {\ab__taufg_ew_seq_cs:n {#1}}
	\seq_if_exist:cTF {\l__ab_tmpb_tl} {
		\int_zero:N \l_tmpa_int
		\seq_map_inline:cn {\l__ab_tmpb_tl} {
			\int_add:Nn \l_tmpa_int {\tl_item:nn {##1} {2}}
		}
		\int_use:N \l_tmpa_int
	}{0}
}

%% Loops
%% Loop over all used exercises
% #1: Code
\cs_new:Npn \ab_foreach_aufg:n #1 {
	\xsim_foreach_exercise_id_type:nn {used} {#1}
}

\cs_new:Npn \__ab_ta_loop_item:nn #1#2 {}
\cs_generate_variant:Nn \__ab_ta_loop_item:nn {xx}

%% Loop over teilaufgaben of an exercise
% #1: ID der Aufgabe
% #2: Code
\cs_new:Npn \ab_foreach_taufg:nn #1#2 {
	\cs_set:Nn \__ab_ta_loop_item:nn {#2}
	%
	\tl_set:Nx \l__ab_tmpa_tl {\ab__aufg_seq_cs:n {#1}}
	\seq_if_exist:cTF {\l__ab_tmpa_tl} {
		\seq_map_inline:cn {\l__ab_tmpa_tl} {
			% TODO: Was ist in item 2 gespeichert?
			\__ab_ta_loop_item:xx
				{\tl_item:Nn {##1} {1}}
				{\tl_item:Nn {##1} {2}}
		}
	}{}
}

\cs_new:Npn \__ab_ew_item_a:nn #1#2 {}
\cs_generate_variant:Nn \__ab_ew_item_a:nn {xx}

%% Loop over erwartungen of an exercise
% #1: ID der Teilaufgabe
% #2: Code
\cs_new:Npn \ab_foreach_erwartung:nn #1#2 {
	\cs_set:Nn \__ab_ew_item_a:nn {#2}
	%
	\tl_set:Nx \l__ab_tmpb_tl {\ab__taufg_ew_seq_cs:n {#1}}
	\seq_if_exist:cTF {\l__ab_tmpb_tl} {
		\seq_map_inline:cn {\l__ab_tmpb_tl} {
			\__ab_ew_item_a:xx
				{\tl_item:Nn {##1} {1}}
				{\tl_item:Nn {##1} {2}}
		}
	}{}
}

\cs_new:Npn \ab__ew_tabellen:n #1 {
	\seq_clear:N \l__ab_tmpa_rows_seq
	\ab__ew_tabellen_erwartungen:n {aufg#1-taufg0}
	% Erwartungen der Aufgabe
	\seq_if_empty:NTF \l__ab_tmpa_seq {} {
		\seq_put_right:Nn \l__ab_tmpa_rows_seq {&}
		\seq_put_right:Nx \l__ab_tmpa_rows_seq {\seq_use:Nn \l__ab_tmpa_seq {\newline}}
		\seq_put_right:Nn \l__ab_tmpa_rows_seq {& \ab_punkte_aufg:n #1 & \\\hline}
	}
	% Erwartungen der Teilaufgaben
	\ab_foreach_taufg:nn {#1} {
		\ab__ew_tabellen_erwartungen:n {##1}
		\seq_put_right:Nn \l__ab_tmpa_rows_seq {##2 &}
		\seq_put_right:Nx \l__ab_tmpa_rows_seq {\seq_use:Nn \l__ab_tmpa_seq {\newline}}
		\seq_put_right:Nn \l__ab_tmpa_rows_seq {& \ab_punkte_taufg:n #1 & \\\hline}
	}
}

\cs_new:Npn \ab__ew_tabellen_erwartungen:n #1 {
	\seq_clear:N \l__ab_tmpa_seq
	\ab_foreach_erwartung:nn {#1} {
		\seq_put_right:Nn \l__ab_tmpa_seq {##1}
	}
}

\DeclareExerciseTableTemplate{tabellen}{
	\underline{Name: \Large\hspace{6cm}}

	\begin{center}\renewcommand{\arraystretch}{1.4}
		%
		\exp_args:NNx \dim_set:Nn \l_tmpa_dim {\textwidth}
		\dim_sub:Nn \l_tmpa_dim {6cm}
		%
		\ab_foreach_aufg:n {
			\ab__ew_tabellen:n {#2}
			\begin{tabularx}{\textwidth}{|M{1cm}|X|M{1.5cm}|M{1.5cm}|} \hline
				% Kopfzeile
				\rowcolor{ab.tabelle.kopf.hg}
				& \textbf{Aufgabe~#2} & \textbf{\ab_punkte_aufg:n #2} & \small  \tabularnewline\hline\hline %
				%
				\seq_use:Nn \l__ab_tmpa_rows_seq {}
			\end{tabularx}\par
		}
	\end{center}

	\begin{flushright}
		\underline{\today, \hspace{3cm}}
	\end{flushright}
}


\DeclareExerciseTableTemplate{tabelle}{\ab@ew@tabelle}
\def\ab@ew@tabelle{
	\underline{Name: \Large\hspace{6cm}}

	\begin{center}
		\renewcommand{\arraystretch}{1.4}
		\exp_args:NNx \dim_set:Nn \l_tmpa_dim {\textwidth}
		\dim_sub:Nn \l_tmpa_dim {6cm}
		\begin{tabularx}{\textwidth}{|M{1cm}|X|M{1.5cm}|M{1.5cm}|} \hline
			% Kopfzeile
			\rowcolor{ab.tabelle.kopf.hg}
			Afg & Die~Schülerin/der~Schüler\dots & mögl.\newline Punkte & \small erreicht \tabularnewline\hline\hline %
			%
			%% Aufgaben, Teilaufgaben und Erwartungen
			% ForEachUsedExerciseByID breaks for tables. Store all used IDs in
			% a (temporary) seq first and loop over that.
			\seq_clear:N \l_tmpa_seq
			\ForEachUsedExerciseByID{\seq_put_right:Nn \l_tmpa_seq {##3}}
			\seq_map_function:NN \l_tmpa_seq \ab@ew@aufgabe
			&&&\\\hline % Somewhere the table ending is messed up. We need to add a blank row for some reason.
			%
			% Fusszeile
			\multicolumn{2}{|r|}{Summe:} & \textbf{\ab_punkte_total:} & \tabularnewline\hline
			% \multicolumn{2}{|r|}{Prozentual:} &  & \tabularnewline\hline
			\multicolumn{2}{|r|}{Note:} & \multicolumn{2}{c|}{} \tabularnewline\hline
		\end{tabularx}
	\end{center}

	\begin{flushright}
		\underline{\today, \hspace{3cm}}
	\end{flushright}
}

\newcommand{\ab@ew@aufgabe}[1]{
	\textbf{#1} &
	\tl_set:Nx \l__ab_tmpa_tl {\ab_taufg_id:nn {#1} {0}}
	\ab_foreach_erwartung:nn {\l__ab_tmpa_tl} {\ab@ew@erwartung{##1}{##2}}
	& \textbf{\PunkteAufgabe{#1}} & \tabularnewline \hline
	%
	% \ab_foreach_taufg:nnn {#1} {\ab@ew@teilaufgabe{#1}{#2}} {\ab@ew@teilaufgabe{#1}{#2}}
	\tl_set:Nx \l__ab_tmpb_tl {\ab__aufg_seq_cs:n {#1}}
	\seq_if_exist:cTF {\l__ab_tmpb_tl} {
		\seq_map_inline:cn {\l__ab_tmpb_tl} {\ab@ew@teilaufgabe##1}
	}{~}
}

\NewDocumentCommand \ab@ew@teilaufgabe {s m m} {
	\int_gzero_new:N \g__ab_taufg_points
	\ref{taufg:#2} &
	\tl_set:Nx \l__ab_tmpa_tl {\ab__taufg_ew_seq_cs:n {#2}}
	\seq_if_exist:cTF {\l__ab_tmpa_tl} {
		%\seq_map_inline:cn {\l__ab_tmpa_tl} {\ab@ew@erwartung##1}
		\int_step_inline:nn {\seq_count:c {\l__ab_tmpa_tl}} {
			\tl_set:Nx \l_tmpb_tl {\seq_item:cn {\l__ab_tmpa_tl} {##1}}
			\int_compare:nNnTF {\seq_count:c {\l__ab_tmpa_tl}}={##1} {
				\ab@ew@erwartung*{\tl_item:Nn \l_tmpb_tl {1}}{\tl_item:Nn \l_tmpb_tl {2}}
			}{
				\ab@ew@erwartung{\tl_item:Nn \l_tmpb_tl {1}}{\tl_item:Nn \l_tmpb_tl {2}}
			}
		}
	}{~}
	& \int_use:N \g__ab_taufg_points & \\ \hline
}
\NewDocumentCommand \ab@ew@erwartung {s m m} {
	\int_gadd:Nn \g__ab_taufg_points {#3}
	#2~(#3P) \IfBooleanF{#1}{\newline}
}


%%%%%%%%%%%%%%%%%%%%%%%%%%%%%%%%
%       Notenverteilung        %
%%%%%%%%%%%%%%%%%%%%%%%%%%%%%%%%
\newcommand{\notenverteilungA}{
	\begin{tabular}{|l|*{\use:n {\prop_count:N \l__xsim_relative_grades_prop}}{c|}} \hline
		\rowcolor{ab.tabelle.kopf.hg}
		\textbf{Note}
		\prop_map_inline:Nn \l__xsim_relative_grades_prop {
			& \prop_get:NnNTF \l__ab_grade_shortnames_prop {##1} \l__ab_tmpa_tl
					{\l__ab_tmpa_tl}
					{##1}
		}
		\\ \hline
		\iftoggle{ab@nv@rel}{\rowcolor{muted.hg}
		\textbf{Schwelle}
		\prop_map_inline:Nn \l__xsim_relative_grades_prop {
			& \xsim_get_relative_grade_goal:nnn {##1}{\,\%}{}
		}
		\\ \hline}{}
		\rowcolor{muted.hg}
		\textbf{Punkte}
		\prop_map_inline:Nn \l__xsim_relative_grades_prop {
			& \xsim_get_grade_goal:nnnn {##1}{points}{}{}% \xsim_get_grade_requirement:nnnn {##1}{points+bonus-points}{}{}
		}
		\\ \hline
	\end{tabular}
}

\cs_new:Npn \ab__verteilung_row:nn #1#2 {
	\prop_get:NnNTF \l__ab_grade_names_prop {#1} \l__ab_tmpa_tl
				{\l__ab_tmpa_tl}
				{#1}
	& \xsim_get_relative_grade_goal:nnn {#1}{\,\%}{}
	& \xsim_get_absolute_grade_requirement:nnn {#1}{}{}
	\\\hline
}
\newcommand{\notenverteilungB}{%
	\rowcolors{2}{muted.hg!10}{muted.hg}
	\begin{tabular}{|l|c|c|} \hline
		\rowcolor{ab.tabelle.kopf.hg}
		Note & Schwelle & Punkte \\\hline
		\prop_map_function:NN \l__xsim_relative_grades_prop \ab__verteilung_row:nn
	\end{tabular}
	\hiderowcolors
}

\newcommand{\notenverteilungC}{%
	\int_set:Nn \l_tmpa_int {\prop_count:N \l__xsim_relative_grades_prop}
	\int_set:Nn \l_tmpa_int {\fp_eval:n { round(\l_tmpa_int/4,0)-1 }}
	\int_zero:N \l_tmpb_int

	\tl_clear_new:N \g__ab_nvc

	\prop_map_inline:Nn \l__xsim_relative_grades_prop {
		\int_compare:nNnTF {0}={\l_tmpb_int} {
			% \tl_put_right:Nn \g__ab_nvc { \begin{minipage}{.20\linewidth} }
			\tl_put_right:Nn \g__ab_nvc { \begin{tabular}{|p{.15\linewidth}|c|}\hline }
			\tl_put_right:Nn \g__ab_nvc { \rowcolor{ab.tabelle.kopf.hg} Note & Punkte \\ \hline }
		}{}

		\tl_put_right:Nn \g__ab_nvc { \prop_get:NnNTF \l__ab_grade_names_prop {##1} \l__ab_tmpa_tl {\l__ab_tmpa_tl} {##1} & }
		\tl_put_right:Nn \g__ab_nvc { \xsim_get_absolute_grade_requirement:nnn {##1}{}{} \\ \hline }

		\int_compare:nNnTF {\l_tmpa_int}={\l_tmpb_int} {
			\tl_put_right:Nn \g__ab_nvc { \end{tabular}\hfill }
			% \tl_put_right:Nn \g__ab_nvc { \end{minipage}\hfill }
			\int_zero:N \l_tmpb_int
		}{
			\int_incr:N \l_tmpb_int
		}
	}

	\rowcolors{2}{muted.hg!10}{muted.hg}
	\tl_use:N \g__ab_nvc
}


\def\ab@ew{\iftoggle{ab@ew@anzeigen}{\Erwartungshorizont}{}}
\AtEndDocument{\ab@ew}

\def\notenverteilung{\notenverteilungA}
\def\erwartungshorizont{\gradingtable}

%%%%%%%%%%%%%%%%%%%%%%%%%%%%%%%%
%          Interface           %
%%%%%%%%%%%%%%%%%%%%%%%%%%%%%%%%
\NewDocumentCommand \PunkteGesamt {} {%
	\ab_punkte_gesamt:
}
\NewDocumentCommand \PunkteAufgabe {m} {%
	\ab_punkte_aufg:n {#1}
}
\NewDocumentCommand \PunkteTeilaufgabe {m m} {%
	\ab_punkte_taufg:nn {#1} {#2}
}
\NewDocumentCommand \PunkteTeilaufgaben {m O{+}} {%
	\seq_clear:N \l__ab_tmpa_seq
	\ab_foreach_taufg:nn {#1} {
		\seq_put_right:NV \l__ab_tmpa_seq {\ab_punkte_taufg:n {##1}}
	}
	\seq_use:Nn \l__ab_tmpa_seq {~#2~}
}
\NewDocumentCommand \PunkteErwartungen {m m O{+}} {%
	\seq_clear:N \l__ab_tmpa_seq
	\tl_set:Nx \l__ab_tmpa_tl {\ab_taufg_id:nn {#1}{#2}}
	\ab_foreach_erwartung:nn \l__ab_tmpa_tl {
		\seq_put_right:Nx \l__ab_tmpa_seq {##2}
	}
	\seq_use:Nn \l__ab_tmpa_seq {~#3~}
}

\newcommand{\NotenverteilungFestlegen}[1]{%
	\expandafter\DeclareGradeDistribution\expandafter{#1}
}
\newcommand{\NotennamenFestlegen}[1]{%
	\seq_clear:N \l__ab_grade_names_prop
	\ab_declare_grade_names:Nn \__ab_declare_grade_name:w {#1}
}

\newcommand{\NotenkurznamenFestlegen}[1]{%
	\seq_clear:N \l__ab_grade_shortnames_prop
	\ab_declare_grade_names:Nn \__ab_declare_grade_shortname:w {#1}
}

\def\Erwartungshorizont{%
	\clearpage\clearpairofpagestyles%
	\ihead{\KopfLinks}\chead{}\ohead{Erwartungshorizont}

	\section*{Erwartungshorizont}
	\erwartungshorizont

	\iftoggle{ab@nv@anzeigen}{
	\begin{center}
		\Notenverteilung
	\end{center}
	}{}%
}

\def\Notenverteilung{%
	\scriptsize\notenverteilung%
}

\newcommand{\ab@clearGradingTable}{
	\seq_clear:N \l__xsim_relative_grades_prop
}
\ExplSyntaxOff


\aboptionen{
	noten=mit-tendenzen
}
