

%% Modul typo

%%%%%%%%%%%%%%%%%%%%%%%%%%%%%%%%
%           Optionen           %
%%%%%%%%%%%%%%%%%%%%%%%%%%%%%%%%
\definecolor{ab.anmerkung.text}{named}{secondary}


\newkomafont{code}{\color{primary}\small\ttfamily}

%%%%%%%%%%%%%%%%%%%%%%%%%%%%%%%%
%            Pakete            %
%%%%%%%%%%%%%%%%%%%%%%%%%%%%%%%%
\RequirePackage{enumitem}
\RequirePackage[normalem]{ulem}

% Zeilenumbruch in \url zulassen
% See: https://tex.stackexchange.com/questions/3033/forcing-linebreaks-in-url
%\g@addto@macro{\UrlBreaks}{\UrlOrds}
\RequirePackage{xurl}

%% Auszeichnungen für Funktionen von Apps
\RequirePackage{menukeys}


%%%%%%%%%%%%%%%%%%%%%%%%%%%%%%%%
%      Textauszeichnungen      %
%%%%%%%%%%%%%%%%%%%%%%%%%%%%%%%%

% Personen
\newcommand*{\person}[1]{\textsc{#1}}

% Operatoren
\newcommand*{\operator}[1]{\textsc{#1}}

% Abkürzungen
\newcommand*{\abbr}[1]{\textsc{#1}}

% Inline code
% \code wird von anderen Paketen auch verwendet
\ifdef{\code}{}{
	\newcommand*{\code}[1]{{\usekomafont{code}#1}}
}

% Ordner, Dateien und Programme

\newcommand*{\programm}[1]{%
	{\usekomafont{code}\textsc{#1}}%
}

\newmenucolortheme{abprimary}{named}{muted.hg}{primary}{primary}[primary][secondary][muted]
\newmenucolortheme{absecondary}{named}{secondary.hg}{primary}{primary}[primary][secondary][muted]
% Standard für Menüs
\changemenucolortheme{menus}{abprimary}

% Ordner-, Dateipfade und Tasten
\newmenustylesimple*{ordnerpfad}[\usekomafont{code}]{\CurrentMenuElement}[/][]{abprimary}
\newmenumacro{\ordner}[/]{ordnerpfad}

\newmenustylesimple*{dateipfad}[\usekomafont{code}]{\CurrentMenuElement}[/][]{abprimary}
\newmenumacro{\datei}[/]{dateipfad}

\copymenustyle{tasten}{shadowedroundedkeys}
\changemenucolortheme{tasten}{abprimary}
\changemenuelement{tasten}{pre}{\usekomafont{code}}
\newmenumacro{\tasten}[+]{tasten}

\nottoggle{ab@farbig}{
	\changemenucolortheme{menus}{gray}
	\changemenucolortheme{tasten}{gray}
	\changemenucolortheme{ordnerpfad}{gray}
	\changemenucolortheme{dateipfad}{gray}
}

%% Anmerkungen
\providecommand{\anmerkung}[3]{\parshape 2 0pt \textwidth 2em \dimexpr\textwidth-2em\relax%
	\textcolor{ab.anmerkung.text}{\textbf{\llap{#1\hspace{1ex}}#2}} {\itshape #3}%
}
% Hinweis
\providecommand{\hinweis}[1]{\anmerkung{}{Hinweis}{#1}}
% Achtung
\providecommand{\warnung}[1]{\anmerkung{}{Achtung}{#1}}
% Tipp
\providecommand{\tipp}[1]{\anmerkung{}{Tipp}{#1}}
% Frage
\providecommand{\frage}[1]{\anmerkung{}{Frage}{#1}}


%%%%%%%%%%%%%%%%%%%%%%%%%%%%%%%%
%            Listen            %
%%%%%%%%%%%%%%%%%%%%%%%%%%%%%%%%
\setlist[1]{labelindent=\parindent}

\newlist{smallitem}{itemize}{3}
\setlist[smallitem]{label=\textbullet,noitemsep}
\newlist{smallenum}{enumerate}{3}
\setlist[smallenum]{label=\arabic*.,noitemsep}
\newlist{smalldescr}{description}{1}
\setlist[smalldescr]{noitemsep}

\newlist{enuma}{enumerate}{1}
\setlist[enuma]{label=\alph*)}
\newlist{enumn}{enumerate}{1}
\setlist[enumn]{label=\arabic*)}
\newlist{enumn*}{enumerate}{1}
\setlist[enumn*]{label=(\arabic*)}
\newlist{enumr}{enumerate}{1}
\setlist[enumr]{label=(\roman*)}
