

%%%%%%%%%%%%%%%%%%%%%%%%%%%%%%%%
%           Optionen           %
%%%%%%%%%%%%%%%%%%%%%%%%%%%%%%%%
\newtoggle{ab@aufg@lsg@seite@anzeigen}

\pgfkeys{
	/absetup/.cd,
		aufgaben/format/.code = \aufgabenformat{#1},
		loesungen/.is choice,
		loesungen/keine/.style 	= {loesungen=ohne},
		loesungen/ohne/.code 		= {\togglefalse{ab@aufg@lsg@seite@anzeigen}\xsimsetup{loesung/print = false,loesung*/print = false}},
		loesungen/seite/.code 	= {\toggletrue{ab@aufg@lsg@seite@anzeigen}\xsimsetup{loesung/print = false,loesung*/print = false}},
		loesungen/folgend/.code = {\togglefalse{ab@aufg@lsg@seite@anzeigen}\xsimsetup{loesung/print = true,loesung*/print = true}},
}

\def\ab@aufg@lsg@seite{\clearpage\pagenumbering{Roman}%
	\chead{Lösungen}%
	\ofoot{}\ifoot{}\cfoot{\Seitenzahl}%
	\printallsolutions%*%
}
\AtEndDocument{\iftoggle{ab@aufg@lsg@seite@anzeigen}{\ab@aufg@lsg@seite}{}}

\definecolor{ab.aufgabe.icons}{named}{secondary}
\definecolor{ab.aufgabe.nummer}{named}{primary}
\definecolor{ab.aufgabe.titel}{named}{black}

\appto\farbschemaAktualisieren{%
	\definecolor{ab.aufgabe.icons}{named}{secondary}%
	\definecolor{ab.aufgabe.nummer}{named}{primary}%
}

\newkomafont{aufgabe}{\usefontofkomafont{subsection}}
\newkomafont{teilaufgabe}{\usefontofkomafont{subsection}}
\newkomafont{loseung}{\usefontofkomafont{subsection}}

%%%%%%%%%%%%%%%%%%%%%%%%%%%%%%%%
%            Pakete            %
%%%%%%%%%%%%%%%%%%%%%%%%%%%%%%%%
\RequirePackage{enumitem}
\RequirePackage{ifthen}

%%%%%%%%%%%%%%%%%%%%%%%%%%%%%%%%
%            Makros            %
%%%%%%%%%%%%%%%%%%%%%%%%%%%%%%%%

%% Horizontale listen z.B. für Unteraufgaben
\RequirePackage{tasks}
\DeclareInstance{tasks}{alphabetize}{default}{}
\NewTasksEnvironment[label=\arabic*)]{tasks*}

%% Paket xsim laden und initialisieren

%\RequirePackage[clear-aux]{xsim}
\RequirePackage[use-files,blank]{xsim}

\xsimsetup{
	file-extension=xsim,
}

\DeclareExerciseProperty{icon}

\loadxsimstyle{abnormal,abkompakt,abohne}

% Aufgabentypen
\DeclareExerciseType{aufgabe}{
	exercise-env = aufgabe,
	solution-env = loesung,
	exercise-name = Aufgabe,
	exercises-name = Aufgaben,
	solution-name = Lösung,
	solutions-name = Lösungen,
	exercise-template = abnormal,
	solution-template = abnormal,
}

\newcounter{teilaufgabe}[aufgabe]
\renewcommand*{\theteilaufgabe}{\alph{teilaufgabe})}

\NewDocumentEnvironment {aufgabenteil} { s o }{%
	\refstepcounter{teilaufgabe}\theteilaufgabe
}{}

\newlist{teilaufgaben}{enumerate}{1}
\setlist[teilaufgaben,1]{label=\alph*),left=0pt,resume}

\NewDocumentCommand \teilaufgabe {s o} {%
	\item
}

\DeclareExerciseType{zusatzaufgabe}{
	exercise-env = aufgabe*,
	solution-env = loesung*,
	counter			 = aufgabe,
	%exercise-name = Zusatzaufgabe,
	%exercises-name = Zusatzaufgaben,
	exercise-name = Aufgabe,
	exercises-name = Aufgaben,
	solution-name = L"osung,
	solutions-name = L"osungen,
	exercise-template = abnormal,
	solution-template = abnormal,
}

\NewDocumentCommand \aufgabenformat { s m } {%
	\xsimsetup{aufgabe/template = #2, aufgabe*/template = #2}
}

\NewDocumentCommand \luecke { s m O{} } {%
	\IfBooleanTF{#1}
		{\framebox[#2]{\strut#3}}
		{\blank[width=#2]{#3}}%
}
