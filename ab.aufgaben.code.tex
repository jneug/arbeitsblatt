

%%%%%%%%%%%%%%%%%%%%%%%%%%%%%%%%
%           Optionen           %
%%%%%%%%%%%%%%%%%%%%%%%%%%%%%%%%
\definecolor{ab.aufgabe.icons}{named}{secondary}
\definecolor{ab.aufgabe.nummer}{named}{primary}
\definecolor{ab.aufgabe.titel}{named}{black}

%% Horizontale listen z.B. für Unteraufgaben
\RequirePackage{tasks}

\DeclareInstance {tasks} {enumerate} {default}{ label = \arabic*) }
\ifdef{\theteilaufgabeni}{
\setlist[teilaufgaben]{
	% Fettdruck von Teilaufgaben entfernen
	label=\alph{teilaufgabeni}),
}}{}

%% Paket xsim laden und initialisieren

%\RequirePackage[clear-aux]{xsim}
\RequirePackage[use-files,blank]{xsim}

\xsimsetup{
	file-extension=xsim,
}

\DeclareExerciseProperty{icon}

\loadxsimstyle{abnormal,abkompakt,abohne}

% Aufgabentypen
\DeclareExerciseType{aufgabe}{
	exercise-env = aufgabe,
	solution-env = loesung,
	exercise-name = Aufgabe,
	exercises-name = Aufgaben,
	solution-name = Lösung,
	solutions-name = Lösungen,
	exercise-template = abnormal,
	solution-template = abnormal,
}

\DeclareExerciseType{teilaufgabe}{
	exercise-env = teilaufgabe,
	solution-env = teilloesung,
	exercise-name = Teilaufgabe,
	exercises-name = Teilaufgaben,
	solution-name = Teillösung,
	solutions-name = Teillösungen,
	exercise-template = abkompakt,
	solution-template = abkompakt,
	within = aufgabe,
	the-counter = \alph{teilaufgabe},
	exercise-heading = \theteilaufgabe)\quad
}

\DeclareExerciseType{zusatzaufgabe}{
	exercise-env = aufgabe*,
	solution-env = loesung*,
	exercise-name = Zusatzaufgabe,
	exercises-name = Zusatzaufgaben,
	solution-name = Lösung,
	solutions-name = Lösungen,
	exercise-template = abnormal,
	solution-template = abnormal,
}
