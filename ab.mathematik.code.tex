%% Fachmodul Mathematik


%%%%%%%%%%%%%%%%%%%%%%%%%%%%%%%%
%            Pakete            %
%%%%%%%%%%%%%%%%%%%%%%%%%%%%%%%%
\RequirePackage{amssymb,amstext,amscd}
\RequirePackage{bm} % bold math
\RequirePackage{xfrac,cancel} % fractions
\iftoggle{ab@farbig}{%
	\renewcommand{\CancelColor}{\color{secondary}}%
}{}

\RequirePackage[notation=german]{skmath}
\RequirePackage{interval}
\intervalconfig{
	separator symbol = {;\,},
}


% Pseudo-Einheiten für Paket siunitx
\DeclareSIUnit\le{LE}
\DeclareSIUnit\fe{FE}
\DeclareSIUnit\ve{VE}

%%%%%%%%%%%%%%%%%%%%%%%%%%%%%%%%
%            Makros            %
%%%%%%%%%%%%%%%%%%%%%%%%%%%%%%%%

% Rest
\def\rest{\,\mathsf{R}}
% Gradzahlen
\def\grad{\ang}
% Prozentzahlen
\def\prozent#1{#1\,\%}



%% Mengen
\newcommand{\Z}{\ensuremath{\mathbb{Z}}}
\newcommand{\N}{\ensuremath{\mathbb{N}}}
\newcommand{\Q}{\ensuremath{\mathbb{Q}}}
\newcommand{\R}{\ensuremath{\mathbb{R}}}


%% Punkte
\NewDocumentCommand \punkt {s o >{\SplitArgument{3}{|}} r()} {%
	\ensuremath{\IfNoValueF{#2}{#2\!}\@punkt #3}
}
\NewDocumentCommand \@punkt {m m m m} {
	\left(#1 \,\middle|\, #2 \IfNoValueF{#3}{\,\middle|\, #3}\right)
}


%% Vektoren
\RenewDocumentCommand \vector {s >{\SplitArgument{3}{|}} r()} {%
	\IfBooleanTF{#1}{\@tvector #2}{\@vector #2}
}
\NewDocumentCommand \@vector {m m m m} {%
	\begin{pmatrix}\IfNoValueTF{#3}
		{ #1 \\ #2 }
		{ #1 \\ #2 \\ #3 }
	\end{pmatrix}
}
\NewDocumentCommand \@tvector {m m m m} {%
	\left(\begin{smallmatrix}\IfNoValueTF{#3}
		{ #1 \\ #2 }
		{ #1 \\ #2 \\ #3 }
	\end{smallmatrix}\right)
}
