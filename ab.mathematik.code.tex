%% Fachmodul Mathematik


%%%%%%%%%%%%%%%%%%%%%%%%%%%%%%%%
%            Pakete            %
%%%%%%%%%%%%%%%%%%%%%%%%%%%%%%%%
\RequirePackage{amssymb,amstext,amscd}
\RequirePackage{bm} % bold math
\RequirePackage{xfrac,cancel} % fractions
\iftoggle{ab@farbig}{%
	\renewcommand{\CancelColor}{\color{secondary}}%
}{}

\RequirePackage[notation=german]{skmath}
\RequirePackage{interval}
\intervalconfig{
	separator symbol = {;\,},
}


% Pseudo-Einheiten für Paket siunitx
\DeclareSIUnit\le{LE}
\DeclareSIUnit\fe{FE}
\DeclareSIUnit\ve{VE}

%%%%%%%%%%%%%%%%%%%%%%%%%%%%%%%%
%            Makros            %
%%%%%%%%%%%%%%%%%%%%%%%%%%%%%%%%

% Rest
\def\rest{\,\mathsf{R}}
% Gradzahlen
\def\grad{\ang}
% Prozentzahlen
\def\prozent#1{#1\,\%}
% Absolutbetrag
\def\abs#1{\left|#1\right|}

%% Mengen
\newcommand{\Z}{\ensuremath{\mathbb{Z}}}
\newcommand{\N}{\ensuremath{\mathbb{N}}}
\newcommand{\Q}{\ensuremath{\mathbb{Q}}}
\newcommand{\R}{\ensuremath{\mathbb{R}}}
