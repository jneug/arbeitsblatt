

%%%%%%%%%%%%%%%%%%%%%%%%%%%%%%%%
%           Optionen           %
%%%%%%%%%%%%%%%%%%%%%%%%%%%%%%%%
\definecolor{ab.muster}{named}{gray}

\usetikzlibrary{patterns}
% \usetikzlibrary{patterns.meta}

%%%%%%%%%%%%%%%%%%%%%%%%%%%%%%%%
%            Makros            %
%%%%%%%%%%%%%%%%%%%%%%%%%%%%%%%%
% liniertes Feld
\NewDocumentCommand \liniert {s O{\linewidth} m O{1cm}} {%
	\begin{tikzpicture}
		\draw(0, 0); % Abstand zum oberen Rand wahren
		\foreach \n in {1,...,#3} % Linien zeichnen
			\draw[color=ab.muster](0, -#4*\n )--(0.99\linewidth, -#4*\n );
	\end{tikzpicture}%
}

% kariertes Feld
\NewDocumentCommand \kariert {s O{0} m O{0.5cm}} {%
	\begin{tikzpicture}
		% Star option
		\IfBooleanTF{#1}{% Größe mit festen Maßen festlegen
			\ifstrequal{#2}{0}{%
				\pgfmathsetlengthmacro{\width}{\linewidth-\pgflinewidth}
			}{%
				\pgfmathsetlengthmacro{\width}{#2-\pgflinewidth}
			}
			\pgfmathsetlengthmacro{\height}{#3}
		}{% Größe als Anzahl Kästchen festlegen
			\ifnumequal{#2}{0}{%
				\pgfmathsetlengthmacro{\width}{
					floor((\linewidth-\pgflinewidth)/#4)*#4
				}
			}{%
				\pgfmathsetlengthmacro{\width}{#2*#4}
			}%
			\pgfmathsetlengthmacro{\height}{#3*#4}
		}
		\draw[color=ab.muster,step=#4]
			(0,0) rectangle (\width,-1*\height)
			(0,0) grid (\width,-1*\height);
	\end{tikzpicture}
}

% Feld mit Milimetermuster
\NewDocumentCommand \millimeter {s O{0} m} {%
	\begin{tikzpicture}%[baseline=(current bounding box).north]
		% Star option
		\IfBooleanTF{#1}{ % Größe mit festen Maßen festlegen
			\ifstrequal{#2}{0}
				{  \pgfmathsetlengthmacro{\width}{\linewidth}  }
				{  \pgfmathsetlengthmacro{\width}{#2}  }
			\pgfmathsetlengthmacro{\height}{-1*#3}
			%
			\pgfmathsetlengthmacro{\anza}{\width-\pgflinewidth}
			\pgfmathsetlengthmacro{\anzb}{\width-(0.5\pgflinewidth)}
			\pgfmathsetlengthmacro{\anzc}{\width-(0.1\pgflinewidth)}
		}{ %
			\ifnumequal{#2}{0}
					{  \pgfmathsetlengthmacro{\width}{\linewidth}  }
					{  \pgfmathsetlengthmacro{\width}{(#2+1)*1cm}  }
			\pgfmathsetlengthmacro{\height}{#3*1cm}
			%
			\pgfmathtruncatemacro{\anza}{(\width-\pgflinewidth)/1cm}
			\pgfmathtruncatemacro{\anzb}{(\width-(0.5\pgflinewidth))/1cm}
			\pgfmathtruncatemacro{\anzc}{(\width-(0.1\pgflinewidth))/1cm}
		}
		%
		\draw[color=ab.muster!50,step=1mm,very thin]
			(0,0) grid (\anzc,\height);
		\draw[color=ab.muster,step=5mm,thin]
			(0,0) grid (\anzb,\height);
		\draw[color=ab.muster,step=10mm,thick]
			(0,0) grid (\anza,\height);
	\end{tikzpicture}
}

% punktiert
\pgfdeclarepatternformonly
	{dotgrid} 							% name
  {\pgfpoint{0pt}{0pt}} % lower left
  {\pgfpoint{5mm}{5mm}}		% upper right
  {\pgfpoint{5mm}{5mm}}		% tile size
  {												% shape description
    \pgfpathcircle{\pgfqpoint{2.5mm}{2.5mm}}{.7pt}
    \pgfusepath{fill}
  }
\NewDocumentCommand \punktiert {s O{0} m} {%
	\begin{tikzpicture}%
		% Star option
		\IfBooleanTF{#1}{% Größe mit festen Maßen festlegen
			\ifstrequal{#2}{0}{%
				\pgfmathsetlengthmacro{\width}{\linewidth}
			}{%
				\pgfmathsetlengthmacro{\width}{#2}
			}
			\pgfmathsetlengthmacro{\height}{#3}
		}{% Größe in Zentimetern
			\ifnumequal{#2}{0}{%
				\pgfmathsetlengthmacro{\width}{\linewidth}
			}{%
				\pgfmathsetlengthmacro{\width}{#2*5mm}
			}%
			\pgfmathsetlengthmacro{\height}{#3*5mm}
		}
		\fill [pattern=dotgrid,pattern color=ab.muster] (0,0) rectangle (\width, \height);
		% \fill [pattern={Dots[distance=5mm]},pattern color=ab.muster] (0,0) rectangle (\width, \height);
	\end{tikzpicture}
}

% hex
% iso
