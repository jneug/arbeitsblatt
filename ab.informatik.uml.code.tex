

%%%%%%%%%%%%%%%%%%%%%%%%%%%%%%%%
%           Optionen           %
%%%%%%%%%%%%%%%%%%%%%%%%%%%%%%%%
\pgfkeys{
	/absetup/.cd,
}

%%%%%%%%%%%%%%%%%%%%%%%%%%%%%%%%
%            Pakete            %
%%%%%%%%%%%%%%%%%%%%%%%%%%%%%%%%
\RequirePackage[school]{pgf-umlcd}


%%%%%%%%%%%%%%%%%%%%%%%%%%%%%%%%
%            Makros            %
%%%%%%%%%%%%%%%%%%%%%%%%%%%%%%%%

% UML (pgf-umlcd)
\renewcommand{\umltextcolor}{black}
\renewcommand{\umldrawcolor}{black}
\renewcommand{\umlfillcolor}{white}

\newenvironment{klassendiagramm}{\begin{tikzpicture}}{\end{tikzpicture}}

%% Notizen/Templates für Inhaltsobjekte!
\newcommand{\contentobject}[2]{\node[draw,dashed,rectangle,fill=white,anchor=west,xshift=-1cm] at (#1.north east) {#2};}

\def\uses{\unidirectionalAssociation}
% Selbstassoziation
% Von: https://tex.stackexchange.com/questions/98021/how-to-extend-pgf-umlcd-with-self-association-connection#98023
\newcommand{\usesself}[3]{%
\coordinate (a) at ($(#1.east) + (0,1)$);
\coordinate (b) at ($(#1.east) + (1,1)$);
\coordinate (d) at ($(#1.east) + (1,-1)$);
\coordinate (e) at ($(#1.east) + (0,-1)$);
\coordinate (t) at ($(#1.east) + (1,0)$);
\coordinate (c) at ($(d)!(b)!(t)$);
  \draw [umlcd style,<-] (a) -- (b)
  node[midway, above]{#2}
  node[midway, below]{#3};
  \draw [umlcd style] (b) -- (c);
  \draw [umlcd style] (c) -- (d);
  \draw [umlcd style] (d) -- (e);
}
