

%%%%%%%%%%%%%%%%%%%%%%%%%%%%%%%%
%           Optionen           %
%%%%%%%%%%%%%%%%%%%%%%%%%%%%%%%%
\newlength{\qrbreite}
\setlength{\qrbreite}{1.5cm}

\pgfkeys{
	/absetup/.cd,
		qr breite/.code={\setlength{\qrbreite}{#1}\qrset{height=#1}},
}

%%%%%%%%%%%%%%%%%%%%%%%%%%%%%%%%
%            Pakete            %
%%%%%%%%%%%%%%%%%%%%%%%%%%%%%%%%
\RequirePackage{qrcode}

\qrset{height=\qrbreite}


%%%%%%%%%%%%%%%%%%%%%%%%%%%%%%%%
%            Makros            %
%%%%%%%%%%%%%%%%%%%%%%%%%%%%%%%%

%% Einfügen eines Qr-Codes als PNG-Bild per \includegraphics.
% Als option wird immer "width=\qrbreite" übergeben.
% [1]: Zusätzliche Optionen für \includegraphics
%  2 : Name des PNG-Bildes (ohne ".png" Endung)
\newcommand*{\includeqrcode}[2][]{\includegraphics[width=\qrbreite,#1]{#2.png}}

%% Einen Link mit QR-Code un d Beschreibung setzen
%  [1]: Breite der minipage (default: .33\textwidth)
%   2 : URL für QR-Code
%  [3]: Angezeigte URL (default: der Wert von 2)
%   4 : Beschreibung
\NewDocumentCommand \qrlink { O{0.33\textwidth} m o m } {%
	\begin{minipage}{#1}
		\begin{center}
			\begin{singlespace}\small
				\qrcode[nolink]{#2}\\[0.75em]
				\textit{#4}\\
				\footnotesize{\IfNoValueTF{#3}{\href{#2}{#2}}{\href{#2}{#3}}}
			\end{singlespace}
		\end{center}
	\end{minipage}
}
