

%%%%%%%%%%%%%%%%%%%%%%%%%%%%%%%%
%           Optionen           %
%%%%%%%%%%%%%%%%%%%%%%%%%%%%%%%%
\pgfkeys{
	/absetup/.cd,
}

\providecolor{ab.diagramme.anteile}{gray}{0.80}
\providecolor{ab.diagramme.winkel}{named}{yellow}
\providecolor{ab.diagramme.winkel2}{named}{red}
\providecolor{ab.diagramme.ball}{gray}{0.20}

%%%%%%%%%%%%%%%%%%%%%%%%%%%%%%%%
%            Pakete            %
%%%%%%%%%%%%%%%%%%%%%%%%%%%%%%%%
\RequirePackage{pgfmath}

%%%%%%%%%%%%%%%%%%%%%%%%%%%%%%%%
%            Makros            %
%%%%%%%%%%%%%%%%%%%%%%%%%%%%%%%%
% Anteile zeichnen
\NewDocumentCommand \tikzRechteck {O{} D(){0,0} m} {\draw[#1] (#2) |- +(#3) |- (#2);}
\NewDocumentCommand \tikzQuadrat {O{} D(){0,0} m} {\draw[#1] (#2) |- +(#3,#3) |- (#2);}

% #1: (x,y) coordinates
% #2: [xdim,ydim] dimensions
% #3: {width} Number of parts in x direction
% #3: {height} Number y parts
% #3: {num} Number of parts to highlight
\newcounter{ngb@anteile}
\NewDocumentCommand \tikzAnteile {s >{\SplitArgument{1}{,}}D(){0,0} >{\SplitArgument{1}{,}}o m m m} {%
	\IfNoValueTF{#3}{%
		\IfBooleanTF{#1}{%
			\@tikzAnteileZeichnen #2 {#4} {#5} {#4} {#5} {#6-(sign(#6)*#4*#5)}
		}{%
			\@tikzAnteileZeichnen #2 {#4} {#5} {#4} {#5} {#6}
		}
	}{%
		\IfBooleanTF{#1}{%
			\@tikzAnteileZeichnen #2 #3 {#4} {#5} {#6-(sign(#6)*#4*#5)}
		}{%
			\@tikzAnteileZeichnen #2 #3 {#4} {#5} {#6}
		}
	}
}
\NewDocumentCommand \@tikzAnteileZeichnen {m m m m m m m}{%
	\pgfmathsetmacro \xMax {#5-1}
	\pgfmathsetmacro \yMax {#6-1}
	\pgfmathsetmacro \xs {#3/#5}
	\pgfmathsetmacro \ys {#4/#6}
	\pgfmathtruncatemacro \d {sign(#7)*-1}
	\pgfmathtruncatemacro \q {#7}
	\setcounter{ngb@anteile}{\q}
	\ifnum\value{ngb@anteile}<0\relax
		\addtocounter{ngb@anteile}{-1}%
	\fi
	\tikzRechteck(#1,#2){#3,#4};
	\foreach \x in {0,...,\xMax} {%
		\foreach \y in {0,...,\yMax} {%
			\ifnum\value{ngb@anteile}<0\relax
				\tikzRechteck(#1+\x*\xs,#2+\y*\ys){\xs,\ys};
			\else
				\tikzRechteck[fill=ab.diagramme.anteile](#1+\x*\xs,#2+\y*\ys){\xs,\ys}
			\fi
			\addtocounter{ngb@anteile}{\d}
		}
	}
}

\NewDocumentCommand \tikzAnteileKreis {s D(){0,0} O{1} m m O{0}} {%
	\pgfmathsetmacro \a {#4}
	\pgfmathsetmacro \b {#5}
	\pgfmathsetmacro \ang {360/#5}
	\pgfmathsetmacro \r {#6}
	\draw (#2) circle (#3);
	\foreach \i in {1,...,\a} {%
		\draw[fill=ab.diagramme.anteile] (#2) -- +({(\i-1)*\ang+\r}:#3) arc ({(\i-1)*\ang+\r}:{\i*\ang+\r}:#3) -- cycle;
	}
	\foreach \i in {\a,...,\b} {%
		\draw (#2) -- +({(\i-1)*\ang+\r}:#3);
	}
}

\NewDocumentCommand \tikzWinkel {s D(){0,0} O{1} m} {%
	\fill[ab.diagramme.winkel,fill opacity=.5,draw=ab.diagramme.winkel2,thick] (#2) -- +(0:{#3/2}) arc (0:#4:{#3/2}) -- cycle;
	\IfBooleanT{#1}{\draw[draw=ab.diagramme.winkel2,thick,->] (#2)+(0:{#3/2}) arc (0:#4:{#3/2});}
	\draw[thick] (#2)+(#4:#3) -- (#2) -- +(#3,0);
}

\NewDocumentCommand \tikzBall {s D(){0,0} m O{ab.diagramme.ball}} {%
	\shade[ball color=#4] (#2) circle (#3);
}
