% !TeX root = ./ab-doc.tex

\part{Dokumenttypen}

Jedes Dokument, das mit \pkg*{arbeitsblatt} gesetzt wird, hat einen \emph{Dokumententyp}.
Der Typ bestimmt vor allem das Layout der Seite und erstellt zum Teil einige Kommandos,
die für einen bestimmten Typ sinnvoll sind. Außerdem laden einige Typen automatisch
\hyperref{part:module}{Module}, die für diesen Typ sinnvoll sind. (Zum Beispiel \module*{aufgaben}
und \module*{bewertung} für \module*{klausur}.)

Der Typ wird beim Laden des Pakets bzw. der Dokumentenklasse mit der Option \option*{typ}
angegeben. Jeder Typ besitzt auch eine Kurzform. Werden mehrere Typen angegeben, wird der
erste Typ geladen und der Rest ignoriert. (Ein Dokument hat genau einen Typ.)

\begin{options}
	\keychoice{typ}{\alleTypen}
	Setzt den Arbeitsblatt-Typ für das Dokument.

	\def\do#1{\opt{#1} Kurzform für \keyis*-{typ}{#1}}
	\docsvlist{arbeitsblatt,klassenarbeit,klausur,checkup,lerntheke}
\end{options}

\section{Typ \texttt{arbeitsblatt}}

Dies ist der Standard, wenn kein anderer Typ gewählt wurde.

\section{Typ \texttt{klassenarbeit}}

\section{Typ \texttt{klausur}}

\section{Typ \texttt{checkup}}

\section{Typ \texttt{lerntheke}}
