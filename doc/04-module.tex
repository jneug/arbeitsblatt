% !TeX root = ./ab-doc.tex

\part{Module}\label{part:module}

\pkg*{arbeitsblatt} stellt die meisten Funktionen in kleinen
\emph{Modulen} bereit. Jedes Modul ist eine Sammlung von
Paketen und Kommandos für einen bestimmten Zweck. Es gibt
beispielsweise Module um Icons einzufügen oder QR-Codes zu
setzen.

Module werden mit \cs*{ladeModule} in der Preambel des Dokuments
geladen. Danach können sie mit \cs*{aboptionen} konfiguriert werden.

\begin{commands}
	\command{ladeModule}[\marg{csvlist}]
	\meta{csvlist} ist eine kommagetrennte Liste von Modulnamen,
	die in das Dokument geladen werden sollen.

	Derzeit stehen diese Module zur Auswahl:
	\def\do#1{\module*{#1}, }
	\dolistloop{\listAlleModule}
\end{commands}

\section{Modul \texttt{typo}}\label{sec:modul-typo}
\begin{itemize}
	\item Benötigte Pakete: \pkg{enumitem}, \pkg{ulem}, \pkg{xurl}
\end{itemize}

Das Modul \module{typo} fügt einige Umgebungen zur  Verbesserung und
der Typographie in das Dokument ein. Vor allem Kommandos zur zusätzlichen
Textauszeichnung und neue Listentypen mit \pkg*{enumitem}.

\begin{commands}
	\command{person}[\marg{text}]
	Auszeichnung von Namen.

	\begin{example}[side-by-side,add-silent-cmds={person}]
  \person{J. Neugebauer}\end{example}

	\command{operator}[\marg{text}]
	Auszeichnung von Operatoren aus dem Lehrplan.

	\begin{example}[side-by-side,add-silent-cmds={operator}]
  \operator{Analysiere}\end{example}
\end{commands}

\subsection{Anmerkungen}\label{subsec:typo-anmerkungen}
Anmerkungen können zur Hervorhebung von Text genutzt werden. Durch das laden einiger
anderer Module (zum Beispiel \module*{icons}), kann die Gestlatung der Anmerkungen
angepasst werden.

% Typo-Anmerkungen wiederherstellen
\loadAnmerkungen{Typo}
\begin{commands}
	\command{tipp}
	\begin{example}[side-by-side,add-silent-cmds={hinweis,tipp,frage,warnung}]
  \tipp{Dies ist ein kurzer Tipp.}\end{example}

	\command{hinweis}[\marg{text}]
	\begin{example}[side-by-side,add-silent-cmds={hinweis,tipp,frage,warnung}]
  \hinweis{Dies ist ein kurzer Hinweis.}\end{example}

	\command{warnung}[\marg{text}]
	\begin{example}[side-by-side,add-silent-cmds={hinweis,tipp,frage,warnung}]
  \warnung{Dies ist eine kurze Warnung.}\end{example}

	\command{frage}[\marg{text}]
	\begin{example}[side-by-side,add-silent-cmds={hinweis,tipp,frage,warnung}]
  \frage{Dies ist eine Frage.}\end{example}

	\command{anmerkung}[\marg{markierung}\marg{name}\marg{text}]
	Dies ist das generische Kommando für Anmerkungen, dass von den konkreten Anmerkungen
	genutzt wird. Durch Neudefinition kann das Aussehen der Anmerkungen angepasst werden.
\end{commands}
% Icons-Anmerkungen wiederherstellen
\loadAnmerkungen{Icon}

\subsection{Listen}\label{subsec:typo-listen}

Mit \pkg*{enumitem} werden einige neue Listenarten definiert. Zu den Standardlisten
werden \texttt{small*}-Varianten erstellt, die den Abstand zwischen Listeneinträgen
entfernen (dies entspricht der \pkg*{enumitem} Option \option*{noitemsep}).

Für Aufzählungen werden neue Listenarten mit verschiedenen Aufzählungsarten erstellt.
Falls diese auch ihne Abstand genutzt werden sollen, kann ihnen als Option
\option*{noitemsep} mitgegeben werden.

\begin{environments}
	\environment{smallitem}[\oarg{opts}]
	\env{itemize} ohne Abstände.

	\environment{smallenum}[\oarg{opts}]
	\env{enumerate} ohne Abstände.

	\environment{smalldescr}[\oarg{opts}]
	\env{description} ohne Abstände.

	\environment{enuma}[\oarg{opts}]
	\env{enumerate} mit Kleinbuschtaben im Format \texttt{a)}.

	\environment{enumn}[\oarg{opts}]
	\env{enumerate} mit Zahlen im Format \texttt{1)}.

	\environment{enumnn}[\oarg{opts}]
	\env{enumerate} mit Zahlen im Format \texttt{(1)}.
\end{environments}

\begin{example}
  \begin{smallitem}
    \item Eins
    \item Zwei
  \end{smallitem}
  \begin{enuma}
    \item Eins
    \item Zwei
  \end{enuma}
\end{example}

\section{Modul \texttt{theme}}\label{sec:modul-theme}
\begin{itemize}
	\item Benötigte Pakete:
\end{itemize}

\section{Modul \texttt{icons}}\label{sec:modul-icons}
\begin{itemize}
	\item Benötigte Pakete: \pkg{fontawesome5}, \pkg{ccicons}
\end{itemize}

Das Modul \module{icons} fügt skalierbare Icons in das Dokument ein.

\begin{options}
	\opt{listen mit icons}
	Ändert das Symbol in \env*{itemize} Listen auf das
	Icon \option*{listen icon} um.

	\keyval-{listen icon}{icon kommando}\Default{\cs*{faAngleRight}}
	Setzt das Icon für \env*{itemize} Listen, falls \option*{listen mit icons}
	gesetzt wurde.

	\keyval-{lizenz icon}{icon kommando}
	Setzt das Icon für die Lizenz, das mit \cs*{LizenzIcon} gesetzt werden kann.
	Das Modul versucht das passende Icon aus der \option*{lizenz} Option auszulesen,
	falls diese gesetzt wurde. Mit dieser Option kann aber ein beliebiges anderes
	der \pkg*{ccicons} oder ein völlig anderes gesetzt werden. So lässt sich auch
	ein Lizenzicon angeben, ohne dass zuvor die Lizenz-Option gesetzt wurde.
\end{options}

\subsection{Zusätzliche Icon-Kommandos}\label{subsec:icons-kommandos}
Das \texttt{icon} Modul erstellt zusätzlich zu den \pkg*{fontawesome5} Kommandos
noch einige zusätzliche, die im schulischen Kontext sinnvoll sind und die einheitliche
Gestaltung erleichtern sollen.

Innerhalb des \pkg*{arbeitsblatt} Pakets werden nur diese Icon-Kommandos verwendet.
\\Durch \cs*{renewcommand}\Marg{\cs*{iconTipp}}\Marg{\cs*{faMagic}} kann so zum Beispiel
das Tipp-Icon im gesamten Dokument geändert werden.

\def\splitIcon#1=#2!{\command{icon#1} \csuse{icon#1} (\cs*{#2})}
\renewcommand*{\do}[1]{\splitIcon#1!}

\begin{multicols}{2}
	\begin{commands}
		\docsvlist{%
			Tipp=faLightbulb[regular],
			Hinweis=faInfo,
			Warnung=faExclamationTriangle,
			Frage=faQuestionCircle[regular],
			Lehrer=faUserTie,Schueler=faUser,
			Mann=faMale,Frau=faFemale,%
			Einzel=faUser,Partner=faUserFriends,Gruppe=faUsers,
			Zustimmung=faThumbsUp,Ablehnung=faThumbsDown,%
			Buch=faBook,Heft=faBookOpen,
			Stift=faPencil*,Rechner=faCalculator,
			Computer=faDesktop,Laptop=faLaptop,
			Smartphone=faMobile*,Tablet=faTablet*,
			Tastatur=faKeyboard[regular],
			Datei=faFile*[regular],Ordner=faFolderOpen[regular],
			Stern=faStar,
			Lachen=faGrinBeam[regular],Froh=faSmile[regular],
			Neutral=faMeh[regular],Traurig=faFrown[regular],
			Boese=faAngry[regular]
		}
		\command{iconSkala} \iconSkala
			\command{iconBox} \iconBox* (\cs*{faSquare[regular]})
			\command{iconBox*} \iconBox* (\cs*{faCheckSquare[regular]})
	\end{commands}
\end{multicols}

\subsection{Anmerkungen}\label{subsec:icons-anmerkungen}
Durch das Laden des \module*{icon} Moduls werden die \hyperref[subsec:typo-anmerkungen]{Anmerkungen}
\cs*{tipp}, \cs*{hinweis}, \cs*{warnung} und \cs*{frage} automatisch
mit Icons angereichert.

\begin{example}
	\tipp{\blindtext}
\end{example}

\subsection{Listen}\label{subsec:icons-listen}

Mit der Option \option*{listen mit icons} werden \env*{itemize} Listen
mit einem Icon als Markierung gesetzt. Wenn vor \module*{icon} das Modul
\module*{typo} geladen wurde, dann wird auch \env*{smallitem} angepasst.

\setlist[itemize]{label=\iconListItem}
\begin{example}
	\begin{itemize}
		\item Eintrag 1
		\item Eintrag 2
		\item Eintrag 3
	\end{itemize}
\end{example}

Das Icon kann entweder in \cs*{aboptionen} mit \key*-{listen icon}{icon} oder
direkt durch \renewcs{iconListItem}{\meta{icon}} geändert werden.

\begin{example}
	\aboptionen{
		listen icon=\faAngleDoubleRight
	}
	\begin{itemize}
		\item Eintrag 1
		\item Eintrag 2
		\renewcommand{\iconListItem}{%
			\textcolor{secondary}{\faHandPointRight[regular]}%
		}
		\item Eintrag 3
		\item Eintrag 4
	\end{itemize}
\end{example}
%% Reset itemize
\setlist[itemize]{label=\textbullet}

\section{Modul \texttt{boxen}}\label{subsec:icons-boxen}
\begin{itemize}
	\item Benötigte Pakete: \pkg{mdframed}, \pkg{awesomebox} (\pkg{fontawesome5})
\end{itemize}

Das Modul \module{boxen} ermöglich den Satz von hervorgehobenen Absätzen, die
durch Rahmen, Hintergrundfarben oder Schattierungen markiert sind. Dazu werden
die Pakete \pkg*{mdframed} und \pkg*{awesomebox} eingebunden. Deren Dokumentationen
enthalten umfassende Informationen zur Verwendung.

\pkg*{awesomebox} bindet auch \pkg*{fontawesome5} ein und nutzt Icons zur Darstellung
der Hervorhebungen. Das Modul \module*{icons} wird dazu nicht benötigt.

\begin{options}
	\opt{anmerkungen als boxen}
	Modifiziert die \hyperref[subsec:typo-anmerkungen]{Anmerkungen} \cs*{tipp}, \cs*{hinweis},
	\cs*{warnung} und \cs*{frage}, sodass sie mit \pkg*{awesomebox} gesetzt werden.

	\important{Wird das Modul \module*{icons} nach \module*{boxen} geladen, dann werden die
		Anmerkungen wieder zurückgesetzt und verwenden Icons. Soll also \option*{anmerkungen als boxen}
		genutzt werden, sollte \module*{boxen} nach \module*{icons} geladen werden.}
\end{options}

\begin{example}[add-silent-cmds={frage,aboptionen}]
	\frage{Wie viele Planeten hat unser Sonnensystem?}
	\aboptionen{anmerkungen als boxen}
	\frage{Wie viele Planeten hat unser Sonnensystem?}
\end{example}

\begin{environments}
	\environment{rahmen}[\oarg{opts}]
	Zeichnet eine \env*{mdframed} um den Inhalt. \meta{opts} sind zusätzliche Optionen
	für \env*{mdframed}.

	\begin{example}[add-silent-envs={rahmen}]
		\begin{rahmen}
			\blindtext
		\end{rahmen}
	\end{example}

	\environment{schattenbox}[\oarg{opts}]
	Zeichnet einen \env*{rahmen} mit zusätzlichem Schatten um den Inhalt. \meta{opts}
	sind zusätzliche Optionen für \env*{mdframed}.

	\begin{example}[add-silent-envs={schattenbox}]
		\begin{schattenbox}
			\blindtext
		\end{schattenbox}
	\end{example}

	\environment{infobox}[\oarg{opts}]
	Zeichnet eine \env*{mdframed} mit Hintergrundfarbe und Schatten um den Inhalt. \meta{opts}
	sind zusätzliche Optionen für \env*{mdframed}. Als Hintergrundfarbe wird \clr*{primary.hg}
	verwendet.

	\begin{example}[add-silent-envs={infobox}]
		\begin{infobox}
			\blindtext
		\end{infobox}
	\end{example}
\end{environments}

\section{Modul \texttt{muster}}\label{sec:modul-muster}
\begin{itemize}
	\item Benötigte Pakete: (\pkg{tikz})
\end{itemize}

Mit dem Modul \module{muster} können Felder verschiedener Papiertypen (liniert,
kariert, Millimeterpapier) gezeichnet werden. Die Felder werden mit \pkg*{tikz}
gezeichnet und können zum Beispiel mit \env*{figure} oder \pkg*{wrapfig} als
Floats gesetzt werden.

\begin{commands}
	\command{liniert}[\sarg\oarg{linienbreite}\marg{anzahl}\oarg{abstand}]
	Zeichnet \meta{anzahl} Linien mit der Länge \meta{linienbreite} und dem
	Abstand \meta{abstand} untereinander.

	\command{kariert}[\sarg\oarg{breite}\marg{höhe}\oarg{größe}]

	\command{millimeter}[\sarg\oarg{breite}\marg{höhe}]

	\command{punktiert}[\sarg\oarg{breite}\marg{höhe}\oarg{abstand}]
\end{commands}
\begin{example}
	\liniert{3}
\end{example}
\begin{example}
	\begin{center}
		\kariert{4}
		\kariert[4]{4}
	\end{center}
\end{example}
\begin{example}
	\millimeter{4}
\end{example}

\section{Modul \texttt{tabellen}}\label{sec:modul-tabellen}
\begin{itemize}
	\item Benötigte Pakete: \pkg{tabularx}, \pkg{longtable}
\end{itemize}

Mit dem Modul \module{tabellen} werden Zusatzpakete und Kommandos für den
Tabellensatz geladen.

\section{Modul \texttt{qrcodes}}\label{sec:modul-qrcodes}
\begin{itemize}
	\item Benötigte Pakete: \pkg{qrcode}
\end{itemize}

Mit dem Modul \module{qrcodes} können Links und Texte (und beliebige
andere Daten) als QR-Codes in das Dokument gesetzt werden.

\begin{options}
	\opt{qr breite}\Default{1.5cm}
	Setzt die Standardgröße für QR-Codes, die mit einem der Kommandos unten
	gesetzt werden. Die Größe kann noch individuell pro Code mit der
	\option*{height} Option des \pkg*{qrcode} Pakets festgelegt werden.

	Die Option setzt die Länge \cs*{qrbreite}, die auch manuell mit
	\cs*{setlength} verändert werden kann.
\end{options}

\begin{commands}
	\command{qrcode}[\oarg{opts}\marg{content}]
	Das normale Kommando aus dem \pkg*{qrcode} Paket. Eine vollständige
	Beschreibung und die Optionen können der Dokumentation des Pakets
	entnommen werden.

	Sinnvolle Optionen sind \key*-{height}{length}, um die Breite festzulegen
	und \option*{nolink}, wenn \meta{content} keine URL ist.

	\important{Das \cs*{qrcode} Kommando hat Probleme mit Sonderzeichen wie
		Umlauten. Wenn auf einem Arbeitsblatt Hinweise als QR-Code gesetzt werden
		sollen, dann ist oftmals das Kommando \cs*{includeqrcode} besser geeignet.}

	\command{includeqrcode}[\oarg{opts}\marg{dateiname}]
	Bindet einen QR-Code mit \cs*{includegraphics} als PNG-Bild ein.
	Das Bild hat den Namen \enquote{\meta{dateiname}.png}. (\meta{dateiname})
	muss also ohne Endung angegeben werden!)
	\meta{opts} wird direkt an \cs*{includegraphics} weitergegeben.

	Der QR-Code
	muss mit anderen Tools generiert werden / worden sein. Siehe
	\autoref{subsec:qrcodes-einbetten} für eine Anleitung, die QR-Codes dynamisch
	mit \pkg{latexmk} zu generieren.

	\command{qrlink}[\oarg{breite}\marg{url}\oarg{kurzurl}\marg{beschreibung}]
	Erstellt einen hervorgehobenen Link zu einer \meta{url} in einer
	\env*{minipage}. Die Breite ist auf 33\,\% von \cs*{textwidth} festglegt,
	kann aber durch \meta{breite} verändert werden.

	Die \meta{url} wird als QR-Code gesetzt und darunter die \meta{beschreibung}.
	Wenn \meta{kurzlink} angegeben wird, wird dieser unterhalb der Beschreibung angezeigt,
	andernfalls wird die vollständige \meta{url} gesetzt.

	Es ist ratsam immer einen \meta{kurzlink} anzugeben, der abgetippt werden kann,
	wenn kein QR-Scanner verfügbar ist. Im Zweifel kann ein Kurzlink mit einem Dienst
	wie \hyperlink{https://www.is.gd}{is.gd} erstellt werden.
\end{commands}

\begin{example}
	\qrcode[height=2.5cm,nolink]{Dieser Text wird als QR-Code gesetzt und kann mit einem QR-Scanner
		gelesen werden. Umlaute sind nicht erlaubt!}
\end{example}
\begin{example}
	\qrlink{https://ngb.schule/wiki}[ngb.schule/wiki]{Informatik-Wiki}
\end{example}

\subsection{QR-Codes einbetten}\label{subsec:qrcodes-einbetten}

Einfache QR-Codes können mit \cs*{qrcode} gesetzt werden. Sollen im QR-Code
komplexere Inhalte (Umlaute, Sonderzeichen, Emojis, ...) codiert werden, dann
muss auf andere Tools zurückgegriffen werden und der Code als Bild eingebettet
werden. \cs*{includeqrcode}\marg{dateiname} hilft dabei.

Es gibt haufenweise Tools, um QR-Codes zu generieren. Wenn \pkg*{latexmk}
verwendet wird, um das Dokument zu kompilieren, kann eine
\href{https://mg.readthedocs.io/latexmk.html#advanced-options}{custom dependency}\footnote{\url{https://mg.readthedocs.io/latexmk.html\#advanced-options}}
(\enquote{eigene Abhängigkeit}) angelegt werden, mit der eine Textdatei automatisch
in ein PNG-Bild transformiert wird.

Zum Beispiel kann das Kommandozeilentool \href{http://fukuchi.org/works/qrencode}{\texttt{qrencode}}\footnote{\url{http://fukuchi.org/works/qrencode}}
zusammen mit dieser Regel in \texttt{.latexmkrc} genutzt werden:

\begin{verbatim}
add_cus_dep( "qr", "png", 0, "compileqr" );
sub compileqr {
  system("cat \"$_[0].qr\" | qrencode -m 0 -o \"$_[0].png\""); }
\end{verbatim}

Der Inhalt des QR-Codes wird dann in der Datei \enquote{\meta{dateiname}.qr} gespeichert.
Wenn \pkg*{latexmk} erkennt, dass keine PNG-Datei mit dem passenden Namen
vorhanden ist, oder die \enquote{.qr}-Datei bearbeitet wurde, wird der
QR-Code beim kompilieren neu erstellt.

Das Beispiel lässt sich leicht für andere Tools anpassen, die über die Kommandozeile
aufgerufen werden können.

\section{Modul \texttt{aufgaben}}\label{sec:modul-aufgaben}
\begin{itemize}
	\item Benötigte Pakete: \pkg{xsim}, \pkg{tasks}
\end{itemize}

\section{Modul \texttt{bewertung}}\label{sec:modul-bewertung}
\begin{itemize}
	\item Benötigte Pakete: \pkg{xsim}
	\item Benötigte Module: \module*{aufgaben}
\end{itemize}

\section{Modul \texttt{varianten}}\label{sec:modul-varianten}

\section{Modul \texttt{meta}}\label{sec:modul-meta}
