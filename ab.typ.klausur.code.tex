

%% Dokumenttyp für Klausuren
\iftoggle{ab@layout}{%
	\RequirePackage[includeheadfoot,top=1.69cm,left=1.69cm,bottom=1.69cm,right=1.69cm,head=30pt]{geometry}
}

%%%%%%%%%%%%%%%%%%%%%%%%%%%%%%%%
%           Optionen           %
%%%%%%%%%%%%%%%%%%%%%%%%%%%%%%%%
\def\ab@kl@dauer{90}
\def\ab@kl@hmft{0}
\def\ab@kl@hmt{90}

\pgfkeys{
	/absetup/.cd,
		dauer/.code=\ab@kl@renewTimes{#1}{\ab@kl@hmft},
		hmft/.code=\ab@kl@renewTimes{\ab@kl@dauer}{#1},
}

\ab@typnameSetzen{Klausur}

%%%%%%%%%%%%%%%%%%%%%%%%%%%%%%%%
%            Pakete            %
%%%%%%%%%%%%%%%%%%%%%%%%%%%%%%%%
\ab@modul@laden{aufgaben}
\ab@modul@laden{bewertung}
\ab@modul@laden{boxen}

% Standard Optionen für Klassenarbeiten
\aboptionen{
	loesungen										= seite,
	erwartungshorizont/anzeigen = true,
	erwartungshorizont/stil     = tabelle, %tabelle2
	noten					 							= kmk-punkte,
	punkte/erwartungen					= true
}

\newcommand{\ab@kl@renewTimes}[2]{%
	\edef\ab@kl@dauer{#1}\edef\ab@kl@hmft{#2}%
	\pgfmathtruncatemacro{\ab@kl@hmt}{\ab@kl@dauer-\ab@kl@hmft}%
}

%%%%%%%%%%%%%%%%%%%%%%%%%%%%%%%%
%            Makros            %
%%%%%%%%%%%%%%%%%%%%%%%%%%%%%%%%

%% Metadaten
\newcommand{\Dauer}{\makeatletter\ab@kl@dauer\makeatother\@\xspace}
\newcommand{\Hmft}{\makeatletter\ab@kl@hmft\makeatother\@\xspace}
\newcommand{\Hmt}{\makeatletter\ab@kl@hmt\makeatother\@\xspace}

%% Seitenlayout
\KOMAoptions{headsepline=.5pt}

\renewcommand{\KopfLinks}{%
	\Fach%
	\ifdefempty{\ab@kurs}{}{\@\xspace\Kurs}%
	\ifdefempty{\ab@kuerzel}{}{\@\xspace(\Kuerzel)}%

	Name:
}
\renewcommand{\KopfMitte}{\relax}
\renewcommand{\KopfRechts}{%
	\Typ%
	\ifdefempty{\ab@nummer}{}{\space Nr. \Nummer}%

	\Datum
}


%% Überschriften etc.
% https://tex.stackexchange.com/questions/223694/how-to-draw-a-text-box-with-shadow-borders-i-have-tried-the-following-but-it-gi#223738
\newcommand{\KlausurTitel}[1][none]{%
\begin{schattenbox}\centering\rmfamily\bfseries
	\Large\Nummer. \Typ (\Dauer\,Minuten)\\
	\large\Fach\ \Kurs\ (\Kuerzel)\ifstrequal{#1}{none}{}{\\
		\textcolor{gray!50!black}{- #1 -}}
\end{schattenbox}}

\newcommand{\vielErfolg}{\begin{flushright}\bfseries Viel Erfolg\end{flushright}}
