
%% -----------------
%%
%% Ein 'theme' zur Dokumentgestaltung (Schriftarten, Farben, ...)
%%
%% -----------------
\setlength{\parindent}{0em}
%\setlength{\parskip}{2ex plus0.5ex minus0.5ex}
\setlength{\parskip}{1ex plus0.5ex minus0.9ex}

% Schriftarten
%% Standard Textkörper (Serifenlos)
\usepackage[sfdefault,light,scaled=.9]{FiraSans}
% \usepackage[sfdefault,light]{FiraSans}
\renewcommand*\oldstylenums[1]{{\firaoldstyle #1}}

%% Überschriften (mit Serifen)
\usepackage{tgschola}
\addtokomafont{disposition}{\rmfamily}

%% Monospace font
%\usepackage{courier}
%\usepackage[light]{noto-mono}
\usepackage[mono,scaled=.9]{inconsolata}

%% Math font
%\usepackage{newtxsf}

% Zeilenabstand anpassen
% https://texwelt.de/wissen/fragen/3/wie-stelle-ich-einen-zeilenabstand-von-15-ein
\usepackage{setspace}
\usepackage{scrhack}
\onehalfspacing

% Farben
%% Allgemein
\definecolor{primary}{rgb}{.16, .227, .655} % {.118, .439, .412}
\definecolor{secondary}{rgb}{.45,.02,.09}
\farbschemaAktualisieren

\definecolor{ab.kopf.text}{gray}{.20}

% Koma Schriften
\addtokomafont{title}{\Large}

%% Informatik
% \ifdefstring{\schule@fach}{Informatik}{
% 	\definecolor{ngb.syntax.hg}{gray}{0.98}
% 	\lstset{
% 		frame=single,
% 		backgroundcolor=\color{ngb.syntax.hg}
% 	}
% }{}
