%%%%%%%%%%%%%%%%%%%%%%%%%%%%%%%%
%           Optionen           %
%%%%%%%%%%%%%%%%%%%%%%%%%%%%%%%%
\pgfkeys{
	/absetup/.cd,
	doku stil/.code=\ab@inf@renewDoku{#1}
}


%%%%%%%%%%%%%%%%%%%%%%%%%%%%%%%%
%            Makros            %
%%%%%%%%%%%%%%%%%%%%%%%%%%%%%%%%
\ab@modul@laden{tabellen}

% Klassendokumentation
\newenvironment{kdokudefault}[1]{%
	\tabularx{\textwidth}{|l|X|}\hline
		%\rowcolor{ab.tabelle.kopf.hg}
		\textbf{Methode} & \textbf{Beschreibung} \\ \hline
	}{\endtabularx}
\newcommand{\mdokudefault}[2]{%\label{methode:#1}
	\textbf{\texttt{#1}} & #2 \tabularnewline \hline
}

% Klassendokumentation wie im Abitur NRW (ab 2021)
\newenvironment{kdokuabineu}[1]{}{\par}
\newcommand{\mdokuabineu}[2]{%\label{methode:#1}
	\colorbox{gray!30}{\parbox{\dimexpr\linewidth-2\fboxsep\relax}{\bfseries\texttt{#1}}}\newline #2
}

\newenvironment{klassendoku}[1]{\kdokudefault{#1}\label{doku:#1}}{\endkdokudefault}
\newcommand{\methodendoku}[2]{\mdokudefault{#1}{#2}}

\newcommand{\ab@inf@renewDoku}[1]{%
	\ifstrequal{#1}{abi21}{%
		\renewenvironment{klassendoku}[1]{\kdokuabineu{##1}\label{doku:##1}}{\endkdokuabineu}%
		\renewcommand{\methodendoku}[2]{\mdokuabineu{##1}{##2}}
	}{
		\renewenvironment{klassendoku}[1]{\kdokudefault{##1}\label{doku:##1}}{\endkdokudefault}%
		\renewcommand{\methodendoku}[2]{\mdokudefault{##1}{##2}}
	}
}
